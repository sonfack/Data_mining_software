\documentclass[12pt]{article}
\usepackage[utf8]{inputenc}
\usepackage[T1]{fontenc}
\usepackage[cyr]{aeguill}
\usepackage[francais]{babel}
\usepackage{adjustbox}
\usepackage{caption}

\begin{document}
	\begin{titlepage}
		\clearpage 
		\title{Les logiciels de Fouille de données}
		\author{SONFACK S. Serge}
		\date{\today}
		\maketitle
		\thispagestyle{empty}
	\end{titlepage}
\pagenumbering{roman}
\newpage
\tableofcontents

\newpage
\listoftables
% page pour le document
\newpage

\section*{Introduction}
	\addcontentsline{toc}{section}{Introduction}
	\pagenumbering{arabic}
	
	\paragraph{ \indent Depuis longtemps, l'Homme est appelé à prendre des décisions ou à faire des choix pour son évolution. Dans la plus part de temps le nombre de critères ou de paramètres sur les quels il doit décider dépassent sont entendement.
	Néanmoins, grâce au numérique et à nos données anciennes ou nouvelles, la fouille de données se veut un outil très important pour soutenir l'Homme lorsqu'il fait face à un grand volume de données et la prise de décisions.
	Dans la suite de notre rapport, nous allons présenter les outils logiciels  utilisés dans le domaine de fouille de données en les classant par type de licence d'utilisation; avec une vue globale ces outils sur un tableau.}
\newpage
	\section{Les logiciels gratuits}
		\paragraph{Les logiciels qui sont téléchargeables gratuitement sur Internet}
		
	\section{Les logiciels libres}
		\paragraph{Les logiciels qui ont une licence libre}

	\section{Les logiciels propriétaires}
		\paragraph{Les logiciels édités par les entreprises}
	\begin{table}[h]
		\caption{\label{tab:table-name}Tableau comparatif}
		\begin{center}	
			\begin{adjustbox}{max width=\textwidth}
				\begin{tabular}{| l | l | l | l | l | l | }		
					\hline
					Logiciels & Commercial & Libre & Monoutilisateur & Clientserveur & Plateforme & Coût\\ \hline
				Clémentine SPSS & Oui & Non & Non & Oui & Mac, Linux, Windows \\ \hline
				Entreprise Miner de SAS & Oui & Non & Non & Oui & Windows, Linux, HP-UX, AIX, Solaris, z/OS \\ \hline
				DATAmaestro & Oui & Non & Non & Oui & Web\\ \hline
				Statisca Data Miner StatSoft & Oui & Non & Non & Oui & Windows \\ \hline 
				Intelligent Miner IBM & Oui & Non & Non & Oui & Unix, Linux, Windows \\ \hline
				
				
				\end{tabular}
			\end{adjustbox}
		\end{center}
	\end{table}

\section*{Conclusion}
	\addcontentsline{toc}{section}{Conclusion}
	\paragraph{Au terme de notre document, nous pouvons dire que le domaine des fouilles de données regorgent un grand nombre de logiciels chacune }
% page pour la bibiographie	
\newpage
\begin{thebibliography}{9}
	\bibitem{Website_wikiversity}
	Wikiversity
	\texttt{https://fr.wikiversity.org/wiki/Datamining/Logiciels}
	
	\bibitem{Wikiwebsite_wikipediat_logiciel} 
	Wikipedia-Logiciel,
	\texttt{https://fr.wikipedia.org/wiki/Logiciel}
	
	\bibitem{Wikiwebsite_spss} 
	Wikipedia-SPSS,
	\texttt{https://fr.wikipedia.org/wiki/SPSS}

	\bibitem{Wikiwebsite_sas} 
	Wikipedia-SAS,	
	\texttt{https://fr.wikipedia.org/wiki/SAS}	
	
\end{thebibliography}

\end{document}